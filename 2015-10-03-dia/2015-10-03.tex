\documentclass[a4paper, landscape, twocolumn, 11pt]{article}
\usepackage[russian]{babel}
\usepackage[utf8]{inputenc}

\usepackage{paralist}

\usepackage{amsmath}


\pagestyle{empty}

\usepackage[margin=1.5truecm, columnsep=1.5truecm, landscape, twocolumn]{geometry}

\begin{document}

\begin{center}

\noindent\texttt{Физтех-лицей \hfill 03 октября 2015}

\vspace{-1ex}
\noindent\rule{0pt}{0pt}\hrulefill\rule{0pt}{0pt}

\bigskip

\textsc{\large Диагностическая работа}

\end{center}


\begin{enumerate}[\bf 1.]

\item В прямоугольном треугольнике $ABC$ гипотенуза $AB$ равна $c$, $\angle ABC = \alpha$. Найти все медианы треугольника.

\item Найдите площадь трапеции с основаниями 18 и 13 и боковыми сторонами 3 и 4.

\item На катетах прямоугольного треугольника как на диаметрах построены окружности. Найдите их общую хорду, если катеты равны 3 и 4.

\item Найдите радиус \textbf{a)}~вписанной \textbf{b)}~описанной окружности треугольника со сторонами 13, 13 и 24.

\item Точки $M$ и $N$ "--- середины сторон $BC$ и $CD$ параллелограмма $ABCD$. $O$ "--- точка пересечения $AM$ и $BN$. Найдите $AO:OM$.

\item Из точки $M$, лежащей вне окружности с центром в точке $O$ и радиусом $R$, проведены касательные $MA$ и $MB$ ($A$ и $B$ "--- точки касания). Оказалось, что отрезок $OM$ делится окружностью пополам. Прямые $OA$ и $MB$ пересекаются в точке $C$. Найдите $OC$.

\item Две окружности с центрами $O_1$ и $O_2$ касаются в точке $C$. Прямая касается этих окружностей в различных точках $A$ и $B$. Найти угол $AO_2B$, если $\tg \angle ABC = \frac12$.

\item Стороны треугольника равны 3 и 6, а угол между ними равен \textbf{a)}~$60^\circ$ \textbf{b*)}~$45^\circ$. Найдите биссектрису треугольника, проведённую из этого угла.

\end{enumerate}

\newpage


\begin{center}

\noindent\texttt{Физтех-лицей \hfill 03 октября 2015}

\vspace{-1ex}
\noindent\rule{0pt}{0pt}\hrulefill\rule{0pt}{0pt}

\bigskip

\textsc{\large Диагностическая работа}

\end{center}


\begin{enumerate}[\bf 1.]

\item В прямоугольном треугольнике $ABC$ гипотенуза $AB$ равна $c$, $\angle ABC = \alpha$. Найти все медианы треугольника.

\item Найдите площадь трапеции с основаниями 18 и 13 и боковыми сторонами 3 и 4.

\item На катетах прямоугольного треугольника как на диаметрах построены окружности. Найдите их общую хорду, если катеты равны 3 и 4.

\item Найдите радиус \textbf{a)}~вписанной \textbf{b)}~описанной окружности треугольника со сторонами 13, 13 и 24.

\item Точки $M$ и $N$ "--- середины сторон $BC$ и $CD$ параллелограмма $ABCD$. $O$ "--- точка пересечения $AM$ и $BN$. Найдите $AO:OM$.

\item Из точки $M$, лежащей вне окружности с центром в точке $O$ и радиусом $R$, проведены касательные $MA$ и $MB$ ($A$ и $B$ "--- точки касания). Оказалось, что отрезок $OM$ делится окружностью пополам. Прямые $OA$ и $MB$ пересекаются в точке $C$. Найдите $OC$.

\item Две окружности с центрами $O_1$ и $O_2$ касаются в точке $C$. Прямая касается этих окружностей в различных точках $A$ и $B$. Найти угол $AO_2B$, если $\tg \angle ABC = \frac12$.

\item Стороны треугольника равны 3 и 6, а угол между ними равен \textbf{a)}~$60^\circ$ \textbf{b*)}~$45^\circ$. Найдите биссектрису треугольника, проведённую из этого угла.

\end{enumerate}


\end{document}