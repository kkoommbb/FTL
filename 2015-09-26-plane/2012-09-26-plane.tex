\documentclass[a4paper, 11pt]{article}
\usepackage[utf8]{inputenc}
\usepackage[russian]{babel}
\usepackage[margin=2truecm]{geometry}

\pagestyle{empty}

\usepackage{paralist}

\begin{document}

\begin{center}
\scshape\Large Планиметрия--1. Свойства биссектрисы
\end{center}

\vfill


\begin{enumerate}[\bf 1.]
\item Пусть в треугольнике $ABC$ $BC=a$, $AC=6$, $AB = c$,
$I$ "--- центр вписанной окружности, $AA_1$ "--- биссектриса 
угла $A$. Докажите, что $AI:IA_1 = (b+c):a$.
\item В прямоугольном треугольнике биссектриса 
острого угла делит катет на отрезки $m$ и $n$. Найдите другой катет
и гипотенузу.
\item Дан треугольник со сторонами 12, 15 и 18. 
Проведена окружность, касающаяся обеих меньших сторон и 
имеющая центр на большей стороне. Найдите отрезки, на 
которые центр окружности делит большую сторону треугольника.
 
\item  Биссектриса треугольника делит одну из его 
 сторон на отрезки с длинами 3 и 5. В каких границах может
 изменяться периметр треугольника?
\item  На гипотенузе $AB$ прямоугольного треугольника
$ABC$ во внешнюю сторону построен квадрат $ABDE$. В каком
отношении делит сторону $DE$ биссектриса угла $C$, если
$AC=1$, $BC=3$?
\item  В выпуклом четырёхугольнике $ABCD$ 
биссектрисы углов $CAD$ и $CBD$ пересекаются на стороне $CD$. 
Докажите, что биссектрисы углов $ACB$ и $ADB$ пересекаются на
стороне $AB$.
\item В треугольнике со сторонами $a$, $b$ и $c$ проведены
биссектрисы, точки пересечения которых с 
противолежащими сторонами служат вершинами второго треугольника.
Найдите отношение площадей этих треугольников.
\item В треугольнике $ABC$ биссектрисы $AD$ и $BE$ 
пересекаются в точке $O$. Найдите отношение площади 
треугольника $ABC$ к площади четырёхугольника $ODCE$, зная, что
$BC = a$, $AC = b$, $AB = c$.
\item В параллелограмме $ABCD$ на сторонах $AB$ и $BC$
выбраны точки $M$ и $N$ соответственно так, что $AM = CN$.
Пусть $Q$ "--- точка пересечения отрезков $AN$ и $CM$. Докажите,
что $DQ$ "--- биссектриса угла $D$.
\item Длины сторон треугольника различны и 
образуют арифметическую прогрессию. Докажите, что прямая, 
проходящая через точку пересечения медиан и центр 
вписанной окружности, параллельна одной из сторон треугольника.
\item Длины сторон треугольника образуют 
арифметическую прогрессию. Докажите, что радиус вписанной
в него окружности втрое меньше одной из высот 
треугольника.
\item Медиана $BK$ и биссектриса $CL$ треугольника $ABC$
пересекаются в точке $P$. Докажите, что 
\[
    \frac{PC}{PL} - \frac{AC}{BC} = 1
\]
\item  Две стороны треугольника равны 10 и 15. 
Докажите, что биссектриса угла между ними не больше 12.
\item Две окружности пересекаются в точках $A$ и $B$.
Прямая, проходящая через точку $A$, вторично пересекает
эти окружности в точках $C$ и $D$, причём точка $A$ лежит
между $C$ и $D$, а хорды $AC$ и $AD$ пропорциональны радиусам
своих окружностей. Докажите, что биссектрисы углов $ADB$
$ACB$ пересекаются на отрезке $AB$.
\item В треугольнике $ABC$ точка $D$ лежит на стороне
$AC$, углы $ABC$ и $BCD$ равны, $AB = CD$, $AE$ "--- биссектриса
угла $A$. Докажите, что отрезки $CD$ и $AB$ параллельны.
\end{enumerate}

\vfill

\noindent\rule{0pt}{0pt}\hrulefill\rule{0pt}{0pt}

\vfill

{\slshape Deadline:} {\bfseries 26 сентября} \hfill Вопросы по задачам присылать на \texttt{feldman@wowmath.ru}

\end{document}