\documentclass[a4paper, landscape, twocolumn, 11pt]{article}
\usepackage[russian]{babel}
\usepackage[utf8]{inputenc}

\usepackage{paralist}

\usepackage{amsmath}


\pagestyle{empty}

\usepackage[margin=1.5truecm, columnsep=1.5truecm, landscape, twocolumn]{geometry}

\begin{document}

\begin{center}

\noindent\texttt{Физтех-лицей \hfill 03 октября 2015}

\vspace{-1ex}
\noindent\rule{0pt}{0pt}\hrulefill\rule{0pt}{0pt}

\bigskip

\textsc{\large Построение сечений}

\end{center}

Три точки построить четвёртую


Метод проекций

\begin{enumerate}[\bf 1.]

\item[\bf *] Октаэдр

\item[\bf **] Треугольник по точкам на сторонах



\end{enumerate}